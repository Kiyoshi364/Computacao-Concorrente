\documentclass[12pt]{article}

\usepackage[brazilian]{babel}

\usepackage{verbatim}
\newenvironment{code}{\verbatim}{\endverbatim}

\usepackage{hyperref}
\hypersetup{
	colorlinks=true,
	linkcolor=cyan,
	filecolor=magenta,      
	urlcolor=blue,
}
\urlstyle{same}

\newcommand{\tb}{\textbf}
\newcommand{\ti}{\textit}
\newcommand{\tu}{\underline}

\begin{document}
\title{Trabalho de Implementação 1 - Computação Concorrente}
\author{Daniel Kiyoshi Hashimoto Vouzella de Andrade}
\date{}
\maketitle

\section{Descrição do problema}

Uma forma de estruturar código de video games
é por meio do ECS (Entity Component System).
Nessa estrutura, são gerados vários sistemas,
que cada um deles simula um comportamento do jogo
podendo ou não depender de outros sistemas
(alguns devem rodar depois de outros).

Dessa forma, pode-se criar um grafo direcionado
aciclico (DAG) e rodar cada sistema na ordem topológica.
Em vários casos, os sistemas têm poucas dependencias
possibilitando percorer o grafo de uma forma
concorrente e eficiente.

O algoritmo sequencial para gerar
uma ordem topológica para um grafo é simples
\href{https://en.wikipedia.org/wiki/Topological_sorting}
{Wikipedia Link}.

\newpage
\section{O Algoritmo Sequencial}

\begin{enumerate}
	\item Gerar uma ``Lista de Graus''.
		Um vetor, de tamanho `número de vértices',
		onde na posição $i$ tem quantos vértices do grafo são adjacentes
		ao vértice $i$.
	\item Cria ``Fila Ordenados''.
	\item Adicionar os vértices iniciais (que tem grau igual a 0)
		em uma ``Fila''.
	\item Enquanto Fila não está vazia:
	\begin{enumerate}
		\item Tirar um vértice ``X'' da Fila.
		\item Adicionar X na Fila Ordenados.
		\item Executar código referente a X.
		\item Para cada vértice ``Y'' que é adjacente a X
			(ou seja, X aponta para Y):
		\begin{enumerate}
			\item Reduzir 1 na posição Y da Lista de Graus.
			\item Se a posição Y == 0, Adicionar Y na Fila.
		\end{enumerate}
	\end{enumerate}
	\item Retorna Fila Ordenados.
\end{enumerate}

\newpage
\section{Projeto do Algoritmo Concorrente}

Alterações e inclusões que vamos fazer uma versão concorrente:
\begin{itemize}
	\item Canais de Comunicação:

		Vamos usar uma Bolsa de Tarefas, uma Lista de Retorno
		e uma de Terminado.

		A Bolsa de Tarefas vai ser um meio de comunicação da
		\verb.main. para operários
		enquanto a Lista de Retorno e de Terminados vão
		ser um meio de comunicação do caminho contrário.

		A Bolsa de Tarefas e de Terminados
		vão ser de um tipo que vou chamar de \verb.mut_arr.:
		\begin{code}
		    typedef struct {
		        int *arr;
		        int len;

		        int i;
		        pthread_mutex_t lock;
		     } mut_arr;
		\end{code}

		Os campos da struct \verb.arr. e \verb.len.
		devem ser escritos somente em tempo de alocação.

		\verb.i. vai ser uma referência ``pública''
		para os trabalhadores, por isso precisa de um mutex.

		\ti{
			Nota: Os} \verb.i.\ti{s para a Bolsa de Tarefas e a
			Lista de Terminados vão ser usados de formas diferentes.
		}

		A Lista de Retorno vai ser um simples vetor.
		O resultado do vértice $i$ pode ser encontrado
		em um de dois lugares: 
		\begin{itemize}
			\item na posição $i$, ou
			\item na posição que $i$ está na Bolsa de Tarefas.
		\end{itemize}

	\item Alterações no Algoritmo da \verb.main.:

		A função \verb.main. vai usar o Algoritmo,
		mas vai ter que mandar trabalhos e esperar
		os operários trabalharem.

		O \ti{novo} Algoritmo vai ser divido em 3 partes:
		\begin{itemize}
			\item Inicialização:
				Passo 1., 2. e 3. do Algoritmo Sequencial
				e inicialização das threads Trabalhadoras.
			\item Cria tarefas:
				Passo 4.a e 4.b do Algoritmo Sequencial
				e ``Escrever vértice X na Bolsa de Tarefas''.
			\item Recebe resultados:
				Para toda tarefa que foi finalizada mas não recebida,
				marcar como recebida e fazer Passo 3.d.
				\ti{
					Junto com a tarefa finalizada,
					será necessário saber qual foi o vértice finalizado.
				}
		\end{itemize}

		O Algoritmo Concorrente fica com essa estrutura:
		\begin{enumerate}
			\item Inicialização.
			\item Enquanto numRecebidos $<$ numVertices:
			\begin{enumerate}
				\item Cria Tarefas.
				\item Recebe resultados.
			\end{enumerate}
		\end{enumerate}

	\item Estrutura dos operários:

		Cada operário vai ficar nesse loop até acabarem as tarefas:
		\begin{enumerate}
			\item Procurar e pegar uma tarefa na Bolsa de Tarefas.
			\item Fazer o trabalho.
			\item Preencher as Listas de Retorno e Terminados.
		\end{enumerate}

\end{itemize}

\newpage
\section{O Algoritmo Concorrente}

Segue uma versão para main:

\begin{enumerate}
	\item Inicializa variáveis globais:
		Bolsa de Tarefas, Lista de Retorno e Lista de Terminados
	\item Gerar uma ``Lista de Graus''.
		Um vetor, de tamanho `número de vértices',
		onde na posição $i$ tem quantos vértices do grafo são adjacentes
		ao vértice $i$.
	\item Criar ``Fila Ordenados''.
	\item Adicionar os vértices iniciais (que tem grau igual a 0)
		na Bolsa de Tarefas.
	\item Criar `número de threads' trabalhadores.
	\item Enquanto Fila Ordenados não está cheia
		(tamanho $<$ `número de vértices'):
	\begin{enumerate}
		\item Tentar tirar um vértice ``X'' da Lista de Terminados.
		\item Se não conseguir, espere um pouco e volte para início do loop.
		\item Adicionar X na Fila Ordenados.
		\item Para cada vértice ``Y'' que é adjacente a X
			(ou seja, X aponta para Y):
		\begin{enumerate}
			\item Reduzir 1 na posição Y da Lista de Graus.
			\item Se a posição Y == 0, Adicionar Y na Bolsa de Tarefas.
		\end{enumerate}
	\end{enumerate}
	\item Sinalizar que os trabalhos acabaram.
	\item Retorna Fila Ordenados.
\end{enumerate}

Versão dos trabalhadores:

\begin{enumerate}
	\item Enquanto existem trabalhos a serem feitos:
	\begin{enumerate}
		\item Tentar pegar atomicamente o vértice ``X''
			da Bolsa de Tarefas.
		\item Se teve sucesso em pegar o vértice:
		\begin{enumerate}
			\item Realizar trabalho associado a X.
			\item Escrever resultado em Lista de Retorno.
			\item Adicionar atomicamente X a Lista de Terminados.
		\end{enumerate}
	\end{enumerate}
\end{enumerate}

\newpage
\section{Testes Realizados}

Como a versão sequencial e a versão concorrente são diferentes,
ou seja, rodar a versão concorrente com uma thread é diferente
de rodar a versão sequencial, foram feitos duas funções
e quando tem ``0 threads'' a main escolhe a versão sequencial.

O arquivo de entrada possui o seguinte formato:
\begin{code}
    NVertices NArestas
    T # Vértices que vértice 0 aponta separados por espaço
    T # Vértices que vértice 1 aponta separados por espaço
    ...
    T # Vértices que vértice NVertices-1 aponta separados por espaço
\end{code}
\verb.T. é o tempo em segundos que o trabalho relacionado ao vértice leva
(argumento para a função sleep).

\ti{
	(Nota:
	Para todos os testes, esse parâmetro será ignorado
	porque a função sleep usada
	não suporta floats nem 0 segundos.
	Esperar 1 segundo 100000 vezes ``a toa'' é um desperdício de tempo
	e também isso daria uma vantagem injusta para as versões concorrentes.)
}

Foi criado um programa para fazer os testes,
ele recebe um número de vértices, arestas, uma seed e um tempo
(se não for oferecido existe um valor padrão).

Existem várias estratégias para formar um DAG,
mas a usada foi uma iterativa,
na qual em cada iteração,
cria-se um vértice que aponta para $X$ já criados.

Foram usados dois tipos de testes
(baseados em sua finalidade).

Os testes de corretude:
\begin{itemize}
	\item in1
	\item in2
	\item in3
	\item auto1
\end{itemize}
Os testes de desempenho:
\begin{itemize}
	\item autoe1-3
	\item autoe1-4
	\item autoe1-5
	\item autoe2-3
	\item autoe2-4
	\item autoe2-5
	\item autoe3-3
	\item autoe3-4
	\item autoe3-5
	\item autoe5-3
	\item autoe5-4
	\item autoe5-5
	\item autoe6-3
	\item autoe6-4
	\item autoe6-5
	\item autoe7-3
	\item autoe7-4
	\item autoe7-5
\end{itemize}
Nomeclatura dos arquivos de teste:
\begin{itemize}
	\item \verb.in[numero].

		Arquivos gerados a mão.
	\item \verb.auto[numero].

		Arquivos gerados pelo programa auxiliar
		para testar o programa auxiliar.
	\item \verb.autoe[exp]-[x].

		Arquivos gerados pelo programa auxiliar
		para testes de desempenho.
		Número de Vértices: $10^{exp}$ \\
		Número de Arestas: $x \cdot 10^{exp}$
\end{itemize}

Foram feitos dois testes,
um com sleep desligado (sem Trabalho),
outro com o sleep ligado (com Trabalho).
Como o sleep minímo é de 1 segundo,
tive que usar grafos pequenos.

Para cada escolha de número de threads
(0, 1, 2, 4)
foi realizado o mesmo teste $5$ vezes
e escolhido o melhor tempo, em segundos.

\newpage
\section{Avaliação de Desempenho}

\begin{center}
	\tb{Tabela de Tempos sem Trabalho}
\begin{tabular}{|c|c||c|c|c|c|}
	\hline
	V & A & 0 & 1 & 2 & 4 \\
	\hline
	      & 3 000 & 0.000053 & 0.000379* & 0.000405 & 0.000593 \\
	1 000 & 4 000 & 0.000084 & 0.000359 & 0.000398 & 0.000605 \\
	      & 5 000 & 0.000071 & 0.000364 & 0.000437 & 0.000637 \\
	\hline
	       & 30 000 & 0.000663 & 0.001832 & 0.001982 & 0.002335 \\
	10 000 & 40 000 & 0.000856 & 0.001674 & 0.002011 & 0.002213 \\
	       & 50 000 & 0.001116 & 0.001727 & 0.002007 & 0.002509 \\
	\hline
	        & 300 000 & 0.012474 & 0.018909 & 0.025661 & 0.027801 \\
	100 000 & 400 000 & 0.016613 & 0.021104 & 0.027023 & 0.029585 \\
	        & 500 000 & 0.018410 & 0.024079 & 0.029371 & 0.031749 \\
	\hline
	          & 3 000 000 & 0.259164 & 0.295392 & 0.364724 & 0.391712 \\
	1 000 000 & 4 000 000 & 0.278765 & 0.319985 & 0.407232 & 0.428438 \\
	          & 5 000 000 & 0.304980 & 0.353866 & 0.437699 & 0.465561 \\
	\hline
	           & 30 000 000 &          &          &          &          \\
	10 000 000 & 40 000 000 & segfault** & segfault** & segfault** & segfault** \\
	           & 50 000 000 &          &          &          &          \\
	\hline
\end{tabular}\end{center}

* Na primeira vez que rodei a bateria o melhor tempo foi 0.000507,
ele estava muito grande e rodei de novo substituindo o resultado.

** Não sei porque deu segmentation fault em todos os casos com
10 000 000 vértices.
A primeira ideia foi porque usava muita memória,
mas o programa auxiliar que gerava os testes
usa a mesma estrutura de dados para guardar o grafo.
Isso não exclui a possibilidade da falta de memória ser a causa
(o algoritmo tem algum overhead de memória),
mas também podem ser ter outros motivos
(algum erro de implementação, por exemplo).

\newpage
\begin{center}
	\tb{Tabela de Acelerações sem Trabalho}
\begin{tabular}{|c|c||c|c|c|}
	\hline
	V & A & 0/1 & 0/2 & 0/4 \\
	\hline
	      & 3 000 & 0.139841 & 0.130864 & 0.089376 \\
	1 000 & 4 000 & 0.233983 & 0.211055 & 0.131868 \\
	      & 5 000 & 0.195054 & 0.162471 & 0.111459 \\
	\hline
	       & 30 000 & 0.361899 & 0.334510 & 0.283940 \\
	10 000 & 40 000 & 0.511350 & 0.425658 & 0.386805 \\
	       & 50 000 & 0.646207 & 0.556053 & 0.444798 \\
	\hline
	        & 300 000 & 0.659685 & 0.486107 & 0.448688 \\
	100 000 & 400 000 & 0.787196 & 0.565625 & 0.523260 \\
	        & 500 000 & 0.764566 & 0.626808 & 0.579860 \\
	\hline
	          & 3 000 000 & 0.877356 & 0.710575 & 0.661618 \\
	1 000 000 & 4 000 000 & 0.871181 & 0.684536 & 0.650654 \\
	          & 5 000 000 & 0.861851 & 0.696780 & 0.669388 \\
	\hline
\end{tabular}\end{center}

\begin{center}
	\tb{Tabela de Tempos com Trabalho}
\begin{tabular}{|c|c||c|c|c|c|}
	\hline
	V & A & 0 & 1 & 2 & 4 \\
	\hline
	   & 30 & 10.003371 & 10.003172 & 6.001523 & 5.003268 \\
	10 & 40 & 10.003835 & 10.002175 & 6.003037 & 5.002508 \\
	   & 50 & 10.003125 & 10.003392 & 5.002935 & 5.006549 \\
	\hline
	    & 300 & 100.054249 & 100.055806 & 51.034063 & 29.033188 \\
	100 & 400 & 100.052372 & 100.046282 & 51.023950 & 28.015765 \\
	    & 500 & 100.049649 & 100.054417 & 51.025788 & 29.224330 \\
	\hline
\end{tabular}\end{center}

\begin{center}
	\tb{Tabela de Acelerações com Trabalho}
\begin{tabular}{|c|c||c|c|c|}
	\hline
	V & A & 0/1 & 0/2 & 0/4 \\
	\hline
	   & 30 & 1.000019 & 1.666805 & 1.999367 \\
	10 & 40 & 1.000165 & 1.666462 & 1.999763 \\
	   & 50 & 0.999973 & 1.999451 & 1.998008 \\
	\hline
	    & 300 & 0.999984 & 1.960538 & 3.446202 \\
	100 & 400 & 0.995914 & 1.960890 & 3.571288 \\
	    & 500 & 0.999952 & 1.960766 & 3.423505 \\
	\hline
\end{tabular}\end{center}

\newpage
\section{Conclusão}

Os resultados do ``sem Trabalho''
aparentemente não foram muito bons.
A ideia é porque não ter nenhum trabalho
leva a todas as threads ficarem ociosas
e só é contado o overhead da programação concorrente
(perda com sincronização e outros).
Entretanto, com o aumento do tamanho dos grafos
a aceleração foi aumentando,
dizendo que realmente existe um ponto em que
esse troca vai valer a pena.

Já os do ``com Trabalho'' seguiram
mais ou menos o esperado:
com mais threads,
consegue executar mais vértices,
e assim percorrer o grafo mais rapidamente,
mas com threads de mais,
falta trabalho e threads começam a ficar ociosas.

Provavelmente esse não foi o melhor jeito de
avaliar o desempenho desse tipo de programa.
Mas não tive outra ideia.
Um dos motivos são os vários fatores que
influenciam no desempenho.
Alguns exemplos são a estrutura do grafo
e complexidade das tarefas.

\end{document}
