\documentclass[12pt]{article}

\usepackage[brazilian]{babel}

\usepackage{verbatim}
\newenvironment{code}{\verbatim}{\endverbatim}

\usepackage{hyperref}
\hypersetup{
	colorlinks=true,
	linkcolor=cyan,
	filecolor=magenta,      
	urlcolor=blue,
}
\urlstyle{same}

\newcommand{\tb}{\textbf}
\newcommand{\ti}{\textit}
\newcommand{\tu}{\underline}

\begin{document}
\title{Trabalho de Implementação 1 - Computação Concorrente}
\author{Daniel Kiyoshi Hashimoto Vouzella de Andrade}
\date{}
\maketitle

\section{Descrição do problema}

Uma forma de estruturar código de video games
é por meio do ECS (Entity Component System).
Nessa estrutura, são gerados vários sistemas,
que cada um deles simula um comportamento do jogo
podendo ou não depender de outros sistemas
(alguns devem rodar depois de outros).

Dessa forma, pode-se criar um grafo direcionado
aciclico e rodar cada sistema na ordem topológica.
Em vários casos, os sistemas têm poucas dependencias
possibilitando percorer o grafo de uma forma
concorrente e eficiente.

O algoritmo sequencial para gerar
uma ordem topológica para um grafo é simples
\href{https://en.wikipedia.org/wiki/Topological_sorting}
{Wikipedia Link}.

\newpage
\section{O Algoritmo Sequencial}

\begin{enumerate}
	\item Gerar uma ``Lista de Graus''.
		Um vetor, de tamanho `número de vértices',
		onde na posição $i$ tem quantos vértices do grafo são adjacentes
		ao vértice $i$.
	\item Cria ``Fila Ordenados''.
	\item Adicionar os vértices iniciais (que tem grau igual a 0)
		em uma ``Fila''.
	\item Enquanto Fila não está vazia:
	\begin{enumerate}
		\item Tirar um vértice ``X'' da Fila.
		\item Adicionar X na Fila Ordenados.
		\item Executar código referente a X.
		\item Para cada vértice ``Y'' que é adjacente a X
			(ou seja, X aponta para Y):
		\begin{enumerate}
			\item Reduzir 1 na posição Y da Lista de Graus.
			\item Se a posição Y == 0, Adicionar Y na Fila.
		\end{enumerate}
	\end{enumerate}
	\item Retorna Fila Ordenados.
\end{enumerate}

\newpage
\section{Projeto do Algoritmo Concorrente}

Alterações e inclusões que vamos fazer uma versão concorrente:
\begin{itemize}
	\item Canais de Comunicação:

		Vamos usar uma Bolsa de Tarefas, uma Lista de Retorno
		e uma de Terminado.

		A Bolsa de Tarefas vai ser um meio de comunicação da
		\verb.main. para operários
		enquanto a Lista de Retorno e de Terminados vão
		ser um meio de comunicação do caminho contrário.

		A Bolsa de Tarefas e de Terminados
		vão ser de um tipo que vou chamar de \verb.mut_arr_t.:
		\begin{code}
		    typedef struct {
		        int *arr;
		        int len;

		        int i;
		        pthread_mutex_t lock;
		     } mut_arr_t;
		\end{code}

		Os campos da struct \verb.arr. e \verb.len.
		devem ser escritos somente em tempo de alocação.

		\verb.i. vai ser uma referência ``pública''
		para os trabalhadores, por isso precisa de um mutex.

		\ti{
			Nota: Os} \verb.i.\ti{s para a Bolsa de Tarefas e a
			Lista de Terminados vão ser usados de formas diferentes.
		}

		A Lista de Retorno vai ser um simples vetor.
		O resultado do vértice $i$ pode ser encontrado
		em um de dois lugares: 
		\begin{itemize}
			\item na posição $i$, ou
			\item na posição que $i$ está na Bolsa de Tarefas.
		\end{itemize}

	\item Alterações no Algoritmo da \verb.main.:

		A função \verb.main. vai usar o Algoritmo,
		mas vai ter que mandar trabalhos e esperar
		os operários trabalharem.

		O \ti{novo} Algoritmo vai ser divido em 3 partes:
		\begin{itemize}
			\item Inicialização:
				Passo 1., 2. e 3. do Algoritmo Sequencial
				e inicialização das threads Trabalhadoras.
			\item Cria tarefas:
				Passo 4.a e 4.b do Algoritmo Sequencial
				e ``Escrever vértice X na Bolsa de Tarefas''.
			\item Recebe resultados:
				Para toda tarefa que foi finalizada mas não recebida,
				marcar como recebida e fazer Passo 3.d.
				\ti{
					Junto com a tarefa finalizada,
					será necessário saber qual foi o vértice finalizado.
				}
		\end{itemize}

		O Algoritmo Concorrente fica com essa estrutura:
		\begin{enumerate}
			\item Inicialização.
			\item Enquanto numRecebidos $<$ numVertices:
			\begin{enumerate}
				\item Cria Tarefas.
				\item Recebe resultados.
			\end{enumerate}
		\end{enumerate}

	\item Estrutura dos operários:

		Cada operário vai ficar nesse loop até acabarem as tarefas:
		\begin{enumerate}
			\item Procurar e pegar uma tarefa na Bolsa de Tarefas.
			\item Fazer o trabalho.
			\item Preencher as Listas de Retorno e Terminados.
		\end{enumerate}

\end{itemize}

\newpage
\section{O Algoritmo Concorrente}

Segue uma versão para main:

\begin{enumerate}
	\item Inicializa variáveis globais:
		Bolsa de Tarefas, Lista de Retorno e Lista de Terminados
	\item Gerar uma ``Lista de Graus''.
		Um vetor, de tamanho `número de vértices',
		onde na posição $i$ tem quantos vértices do grafo são adjacentes
		ao vértice $i$.
	\item Criar ``Fila Ordenados''.
	\item Adicionar os vértices iniciais (que tem grau igual a 0)
		na Bolsa de Tarefas.
	\item Criar `número de threads' trabalhadores.
	\item Enquanto Fila Ordenados não está cheia
		(tamanho $<$ `número de vértices'):
	\begin{enumerate}
		\item Enquanto Lista de Terminados não está vazia:
		\begin{enumerate}
			\item Tirar um vértice ``X'' da Lista de Terminados.
			\item Adicionar X na Fila Ordenados.
			\item Para cada vértice ``Y'' que é adjacente a X
				(ou seja, X aponta para Y):
			\begin{enumerate}
				\item Reduzir 1 na posição Y da Lista de Graus.
				\item Se a posição Y == 0, Adicionar Y na Bolsa de Tarefas.
			\end{enumerate}
		\end{enumerate}
	\end{enumerate}
	\item Sinalizar que os trabalhos acabaram.
	\item Retorna Fila Ordenados.
\end{enumerate}

Versão dos trabalhadores:

\begin{enumerate}
	\item Enquanto existem trabalhos a serem feitos:
	\begin{enumerate}
		\item Tentar pegar atomicamente o vértice ``X''
			da Bolsa de Tarefas.
		\item Se teve sucesso em pegar o vértice:
		\begin{enumerate}
			\item Realizar trabalho associado a X.
			\item Escrever resultado em Lista de Retorno.
			\item Adicionar atomicamente X a Lista de Terminados.
		\end{enumerate}
	\end{enumerate}
\end{enumerate}

\newpage
\section{Testes Realizados}
%TODO

\newpage
\section{Avaliação de Desempenho}
%TODO

\end{document}
