\documentclass[12pt]{article}

\usepackage[brazilian]{babel}

\begin{document}

\section*{Resultados}

\begin{center}
    \begin{tabular}{|c|c|c|c|}
        \hline
        \multicolumn{4}{|c|}{Tempos em Segundos} \\
        \hline
        Dimensão / Threads & 1 & 2 & 4 \\
        \hline
         500 &  0.524514 &  0.291099 &  0.265471 \\
        1000 &  5.232618 &  3.065325 &  2.927719 \\
        2000 & 51.285087 & 31.529268 & 28.964172 \\
        \hline
    \end{tabular}
    
    \ \\[3ex]
    
    \begin{tabular}{|c|c|c|c|}
        \hline
        \multicolumn{3}{|c|}{Acelerações} \\
        \hline
        Dimensão / Threads & 1/2 & 1/4 \\
        \hline
         500 & 1.801840 & 1.975786 \\
        1000 & 1.707035 & 1.787267 \\
        2000 & 1.626586 & 1.770638 \\
        \hline
    \end{tabular}
\end{center}

\section*{Conclusões}

\ 

Os tempos são o esperado, mas as acelerações não.

As acelerações diminuem com o aumento da dimensão;
isso é estranho porque,
a menos que tenha algo incomum relacionado à arquitetura do computador
(menos cores que threads, problemas de cache, ...),
as acelerações deveriam ser crescentes.
Entretanto, elas permanecem maior que 1,
indicando que existem algum ganho de performance.

\end{document}
